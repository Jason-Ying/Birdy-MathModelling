The essential requirement to form an alliance is that it can bring enough extra benefits. From Figure \ref{fig:mechanism}, we can know that the source of the extra benefits is the non-stopping needs of customers and the lure from the discounts brought about by the reward points.

Suppose the alliance (if formed) is formed by \A\ and \B.

\begin{enumerate}
	\itembf{Relative business size}
	
	If \A\ is famous while \B\ has no fame at all, customers will never purchase product from B and in turn get \RP{A}, or go to \B\ with the reward points from \RP{B}. Only when the allies bear similar fame will the alliance be formed.
	
	\itembf{Product type}
	
	Since the essential of the system is mutual advertisements, the allies should not block each other's business and maximize the stimulating effect of purchasing in its ally's store.
	
	\begin{enumerate}
		\itembf{Difference}
		
		As a business alliance, the allies should not block each other's business, meaning that the allies should not sell too same kinds of products.
		
		\itembf{Correlation}
		
		Since the essential of the system is mutual advertisements, the allies should maximize the stimulating effect of purchasing in its ally's store, so \A's products should mostly fit in the potential shopping list of \B's customers (and vice versa).
	\end{enumerate}
	
	\itembf{Distance between nearest offline stores}
	
	In the case of offline stores, since (suppose) the customers receive the recommendation (reward points) from \A, they would go back home and go for \B\ on another day, the nearest offline stores should be close enough for the customers to think the extra distance worth the discount.
\end{enumerate}


