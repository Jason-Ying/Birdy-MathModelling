\hspace{\parindent} A loyalty points system is a decentralized platform based on blockchain, through which merchants can obtain benefits. By developing mathematical models, the influence of various factors on the yearly revenue of merchants in one-to-one alliances and many-to-many alliances are quantified and calculated, the way of distributing points optimized. The research on this problem provides suggestion and references for the future development of similar system and promote merchants to greater development.

In our first model, in order to determine the change in business yearly revenue for a merchant from a one-to-one alliance, we need to abstract the behavior of the alliance and find out the factors that affect the formation of alliance, so as to develop our model. Through programming, we develop the Revenue Change Model(1-1), and divide the customers into three groups by the merchants they have gone shopping for based on the data before and after joining the alliance. Then we calculate the benefit the alliance bring about to merchants. In addition, we calculate the losses from issuing \RPd s, and then calculate the annual growth rate of merchants based on these two results. By inputting the data, merchants can use the output to estimate their yearly revenue growth.

To measure the overall positive impact of a many-to-many alliance to an ally, we create the second model --- the Revenue Change Model (m-m). Through analysis, we find the key factors that affect the impact. By using data fitting and partial differential equations, we get the functional expression of the factors and conclude the expression to calculate the impact. The expressions can be found in the main text, which can determine the optimal number of merchants to form such an alliance with different business scales. Considering that different merchants bring different benefits to the alliance, we form a mathematical model and get a more reasonable way of allocating reward points. It is worth mentioning that in this model, we quantify the factors well and make it as close to the facts as possible, which makes the expression stricter.

We conduct a comprehensive sensitivity analysis on the model and analyze the arguments' influence on the model results. Most of the variables are proved to be insensitive, which shows that our model is reliable.

The highlights of this paper include: our Revenue Change Model(1-1) only depends on the data input by merchants, so it's rather independent, which means it's well-adapted and is easy to promote; at the same time, it can use data for a short period of time to estimate yearly revenue growth, so it's efficient. And our Revenue Change Model(m-m) is comprehensive. It covers many variables related to yearly revenue and is well quantified, which makes our expression stricter. In conclusion, we comprehensively analyze and determine the growth of yearly revenue after merchants join in one-to-one or many-to-many alliances, which has practical significance.