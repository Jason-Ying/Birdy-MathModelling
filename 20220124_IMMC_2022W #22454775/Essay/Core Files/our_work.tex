\begin{enumerate}
	\itembf{Create a model to calculate the change in business yearly revenue for a merchant in a one-to-one alliance:}
		
	\begin{enumerate}[label=(\alph*)]
	\item Abstract the behavior of a one-on-one alliance.
	\item Find the key factors that affect the operation.
	\item Write a program (One-to-One Alliance Revenue Estimation Algorithm).
	
	Our model needs the data of merchants before and after joining the alliance. Using these data, we divide the customers into three groups by the merchants they have gone shopping for before the form of the alliance. Then we calculate respectively the additional benefit brought by buying a specific type of product and the losses from issuing \RPd s. The model provides merchants with their revenue growth rate.
	\end{enumerate}

	\itembf{Create a model to measure the overall positive impact of a many-to-many alliance to an ally:}

	\begin{enumerate}[label=(\alph*)]
	\item List the factors that affect $I$ and list the expressions to calculate $I$, the factors including $I_A$, $I_D$, $\phi$ and $RC$.
	\item Use partial differential equations to get the functional expression of $RC$.
	\item Get the mathematical expression of $I_A$, finding that the key factors affecting it include $n$.
	\item Find out how $I_D$ and $\phi$ relate to the factors and get the the functional expression of it.
	\item Explore the relationship between product types of every two allies in a many-to-many alliance and its impact on the alliance efficiency.

	By applying to the final expression, we can calculate $I$. It measures the overall positive impact on the merchant, which is related to how well a many-to-many alliance is formed. And the optimal number of merchants to form such an alliance can also be calculated, it changes as $s$ changes, indicating that the optimal number of merchants varies to the allies.
	\end{enumerate}

	\itembf{Conduct sensitivity analyses of our model:}
		
	To assess the robustness of our model, we conduct sensitivity analyses on $n$, $\phi$ and $s$. From the final result, we discover that the change of yearly revenue of each merchant won't affect the alliance model.

	\itembf{Find the optimal distribution of \RPd :}
	Since different merchants in an alliance bring different benefits to the alliance, they should be allocated different amount of \RPd\ according to the conditions. What we need to do is to provide a more reasonable distribution for \RPd\ through modeling.

\end{enumerate}